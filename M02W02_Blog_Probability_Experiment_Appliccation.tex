\documentclass{article}
\usepackage[utf8]{inputenc}
\usepackage{amsmath}
\usepackage{amsfonts}
\usepackage{amssymb}
\usepackage{geometry}
\usepackage{listings}
\usepackage{xcolor}
\usepackage{caption}
\usepackage{enumitem}
\geometry{a4paper, margin=1in}

\definecolor{codegreen}{rgb}{0,0.6,0}
\definecolor{codegray}{rgb}{0.5,0.5,0.5}
\definecolor{codepurple}{rgb}{0.58,0,0.82}
\definecolor{backcolor}{rgb}{0.95,0.95,0.92}

\lstdefinestyle{mystyle}{
    backgroundcolor=\color{backcolor},
    commentstyle=\color{codegreen},
    keywordstyle=\color{magenta},
    numberstyle=\tiny\color{codegray},
    stringstyle=\color{codepurple},
    basicstyle=\footnotesize,
    breakatwhitespace=false,
    breaklines=true,
    captionpos=b,
    keepspaces=true,
    numbers=left,
    numbersep=5pt,
    showspaces=false,
    showstringspaces=false,
    showtabs=false,
    tabsize=2
}

\lstset{style=mystyle}

\title{Basic Probability and Its Applications in AI}
\author{Bui Van Tai}
\date{}
\begin{document}

\maketitle
In this section, we'll examine several examples of probability applied to different types of data.

\section*{3.1 Naive Bayes Classifier for Discrete Random Variables}
\subsection*{Problem Example: Predicting Whether to Play Tennis}

We are given a small dataset to predict whether a person will play tennis based on four \textbf{discrete features} (a "naive" hypothesis, yet effective): \textit{Outlook}, \textit{Temperature}, \textit{Humidity} and \textit{Wind}.

\subsection*{Dataset (5 Samples)}

\begin{center}
\begin{tabular}{|c|c|c|c|c|}
\hline
Outlook & Temp & Humidity & Wind & Play \\
\hline
Sunny & Hot & High & Weak & No \\
Overcast & Hot & High & Weak & Yes \\
Rain & Mild & High & Weak & Yes \\
Sunny & Cool & Normal & Weak & Yes \\
Rain & Mild & High & Strong & No \\
Sunny & Cool & High & Strong & ??? \\
\hline
\end{tabular}
\end{center}

\subsection*{Step 1: Bayes's Theorem Overview}
We want to compute:
$$
P(\text{Play=Yes} \mid X), \quad P(\text{Play=No} \mid X)
$$
where $$X = (\text{Outlook = Sunny}, \text{Temp = Cool}, \text{Humidity = High}, \text{Wind = Strong})$$
According to Bayes's Theorem:

$$
P(\text{Class} \mid X) = \frac{P(X \mid \text{Class}) \cdot P(\text{Class})}{P(X)}
$$
Since $P(X)$ is constant for both classes, we compare:
$$
P(X \mid \text{Yes}) \cdot P(\text{Yes}) \quad \text{and} \quad P(X \mid \text{No}) \cdot P(\text{No})
$$

\subsection*{Step 1: Compute Prior Probabilites}
From the dataset: 

$$
P(\text{Yes}) = \frac{3}{5} = 0.6
\quad , \quad
P(\text{No}) = \frac{2}{5} = 0.4
$$

\subsection*{Step 2: Likelihoods (with Laplace Smoothing)}

Apply Laplace smoothing (add-one) to avoid zero probabilities
$$
P(x_i \mid \text{Class}) = \frac{\text{count}(x_i, \text{Class}) + 1}{\text{count}(\text{Class}) + N_i}
$$

Where $N_i$ is the number of possible values for feature $i$.

- Outlook: 3 values (Sunny, Overcast, Rain)

- Temperature: 3 values (Hot, Mild, Cool)

- Humidity: 2 values (High, Normal)

- Wind: 2 values (Weak, Strong)

\subsection*{Step 3: Compute Likelihoods}

\subsubsection*{For Class = Yes (3 samples)}

$$
P(X \mid \text{Yes}) = P(\text{Sunny} \mid \text{Yes}) \cdot P(\text{Cool} \mid \text{Yes}) \cdot P(\text{High} \mid \text{Yes}) \cdot P(\text{Strong} \mid \text{Yes})
$$

$$
= \frac{1+1}{3+3} \cdot \frac{1+1}{3+3} \cdot \frac{2+1}{3+2} \cdot \frac{0+1}{3+2}
= \frac{2}{6} \cdot \frac{2}{6} \cdot \frac{3}{5} \cdot \frac{1}{5} = \frac{12}{900}
$$

$$
P(X \mid \text{Yes}) \cdot P(\text{Yes}) = \frac{12}{900} \cdot 0.6 = \frac{7.2}{900} \approx 0.008
$$

\subsubsection*{For Class = No (2 samples)}

$$
P(X \mid \text{No}) = P(\text{Sunny} \mid \text{No}) \cdot P(\text{Cool} \mid \text{No}) \cdot P(\text{High} \mid \text{No}) \cdot P(\text{Strong} \mid \text{No})
$$

$$
= \frac{1+1}{2+3} \cdot \frac{0+1}{2+3} \cdot \frac{2+1}{2+2} \cdot \frac{1+1}{2+2}
= \frac{2}{5} \cdot \frac{1}{5} \cdot \frac{3}{4} \cdot \frac{2}{4} = \frac{12}{400}
$$

$$
P(X \mid \text{No}) \cdot P(\text{No}) = \frac{12}{400} \cdot 0.4 = \frac{4.8}{400} = 0.012
$$

\subsection*{Conclusion}

Since:
$$
P(\text{No} \mid X) > P(\text{Yes} \mid X)
$$
We conclude that the person will \textbf{not play tennis}.

\section*{Python Code}
\begin{lstlisting}[language=Python, caption={Naive Bayes Classifier}]
# Training data (5 samples)
data = [
    ['Sunny', 'Hot', 'High', 'Weak', 'No'],
    ['Overcast', 'Hot', 'High', 'Weak', 'Yes'],
    ['Rain', 'Mild', 'High', 'Weak', 'Yes'],
    ['Sunny', 'Cool', 'Normal', 'Weak', 'Yes'],
    ['Rain', 'Mild', 'High', 'Strong', 'No']
]

# Input to predict
X = ['Sunny', 'Cool', 'High', 'Strong']

# Extract labels and features
labels = [row[-1] for row in data]
features = [row[:-1] for row in data]

# Calculate prior probability P(Class)
def prior_prob(class_label):
    return sum(1 for label in labels if label == class_label) / len(labels)

# Calculate conditional probability P(x_i | Class) with Laplace smoothing
def cond_prob(feature_idx, feature_val, class_label):
    count = 0
    total = 0
    unique_vals = set(row[feature_idx] for row in features)

    for i, row in enumerate(features):
        if labels[i] == class_label:
            total += 1
            if row[feature_idx] == feature_val:
                count += 1

    # Apply Laplace smoothing
    return (count + 1) / (total + len(unique_vals))

# Compute P(X | Yes) * P(Yes)
yes_prob = prior_prob('Yes')
for i in range(len(X)):
    yes_prob *= cond_prob(i, X[i], 'Yes')

# Compute P(X | No) * P(No)
no_prob = prior_prob('No')
for i in range(len(X)):
    no_prob *= cond_prob(i, X[i], 'No')

# Final prediction
print(f"P(X | Yes) * P(Yes) = {yes_prob}")
print(f"P(X | No) * P(No) = {no_prob}")
if yes_prob > no_prob:
    print("=> Prediction: Play")
else:
    print("=> Prediction: Do not play")
\end{lstlisting}


$Output$

\begin{lstlisting}
P(X | Yes) * P(Yes) = 0.008
P(X | No) * P(No) = 0.012
=> Prediction: Do not play
\end{lstlisting}


\section*{3.2 Naive Bayes for Continuous Random Variable}

Unlike discrete variables, which can take only a finite or countable set of distinct values, \textit{continuous variables} can take infinite number of possible values (height, weight, temperature,...).

This example shows how to apply Navie Bayes Classifier to continuous data using Gaussian distribution.

We are given 5 BMI values and their corresponding class labels:

\begin{center}
\begin{tabular}{|c|c|}
\hline
BMI (x) & Class (y) \\
\hline
18 & 0 (Healthy) \\
20 & 0 \\
22 & 0 \\
26 & 1 (Sick) \\
28 & 1 \\
24 & ??? \\
\hline
\end{tabular}
\end{center}

We want to classify a new sample with BMI=24.

\subsection*{Step 1: Compute mean and variance for each class}
\subsubsection*{For class $C_0$ (Healthy):}

General formulas:
$$
\mu _0 = \frac{1}{n_0} \sum_{i=1}^{n_0} x_i
\quad  and \quad 
\sigma_0^2 = \frac{1}{n_0} \sum_{i=1}^{n_0} (x_i - \mu  _0)^2
$$

Apply to data: [18, 20, 22]

$$
\mu_0 = \frac{18 + 20 + 22}{3} = 20
\quad
$$
$$
\sigma _0^2 = \frac{(18 - 20)^2 + (20 - 20)^2 + (22 - 20)^2}{3} = \frac{8}{3} \approx 2.67
\quad
$$


$$
\sigma_0 = \sqrt{2.67} \approx 1.63
$$

\subsubsection*{For Class $ C_1 $ (Sick):}
Apply to data: [26, 28]

$$
\mu _1 = \frac{26 + 28}{2} = 27
\quad , \quad
\sigma _1^2 = \frac{(26 - 27)^2 + (28 - 27)^2}{2} = 1
\quad , \quad
\sigma _1 = \sqrt{1} = 1
$$

\subsection*{Step 2: Compute Pior Probabilities}

$$
P(C_0) = \frac{n_0}{n} = \frac{3}{5} = 0.6
$$

$$
P(C_1) = \frac{n_1}{n} = \frac{2}{5} = 0.4
$$

\subsection*{Step 3: Use Gaussian Probability Density Function}

General formula:
$$
P(x|C_k) = \frac{1}{\sqrt{2\pi \sigma_k^2}} \cdot e^{-\frac{(x - \mu_k)^2}{2\sigma_k^2}}
$$

\subsubsection*{For class 0:}

$$
P(x=24|C_0) = \frac{1}{\sqrt{2\pi \cdot 2.67}} \cdot e^{-\frac{(24 - 20)^2}{2 \cdot 2.67}} \approx 0.0122
$$

$$
P(C_0|x=24) \propto 0.0122 \cdot 0.6 = 0.0073
$$


\subsubsection*{For class 1:}
$$
P(x=24|C_1) = \frac{1}{\sqrt{2\pi \cdot 1}} \cdot e^{-\frac{(24 - 27)^2}{2 \cdot 1}} \approx 0.0044
$$

$$
P(C_1|x=24) \propto 0.0044 \cdot 0.4 = 0.0018
$$

\subsection*{Step 4: Final Prediction}

$$
P(C_0|x=24) \propto 0.0073 \quad
P(C_1|x=24) \propto 0.0018
\Rightarrow \text{Predict Class 0 (Healthy)}
$$

\section*{Python Code}
We use scikit-learn library, which provides efficient and easy-to-use tools for implementing Naive Bayes Classifier models, making the process simpler.
\begin{lstlisting}[language=Python, caption={NBC For Continuous Random Variable}]

from sklearn.naive_bayes import GaussianNB
import numpy as np

# Input data
X = np.array([[18], [20], [22], [26], [28]])
y = np.array([0, 0, 0, 1, 1])

# Train model
model = GaussianNB()
model.fit(X, y)

# Predict for BMI = 24
x_test = np.array([[24]])
predicted_class = model.predict(x_test)
print("Predicted class:", predicted_class[0])

\end{lstlisting}

\begin{lstlisting}
Predicted class: 0 
\end{lstlisting}


Bayes'theorem demonstrates its wide-ranging and adaptable applicability in various problem domains and data types.

\end{document}